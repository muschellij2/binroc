\documentclass[12pt]{article}
\usepackage{amsmath}
%\usepackage{graphicx,psfrag,epsf}
\RequirePackage{graphicx,color,ae,fancyvrb}

% \usepackage{psfrag,epsf}
\usepackage{enumerate}
\usepackage{natbib}
\usepackage{hyperref} % not crucial - just used below for the URL 


%\pdfminorversion=4
% NOTE: To produce blinded version, replace "0" with "1" below.
\newcommand{\blind}{0}

% DON'T change margins - should be 1 inch all around.
\addtolength{\oddsidemargin}{-.5in}%
\addtolength{\evensidemargin}{-.5in}%
\addtolength{\textwidth}{1in}%
\addtolength{\textheight}{1.3in}%
\addtolength{\topmargin}{-.8in}%

% Pandoc header
\usepackage{float}
\let\origfigure\figure
\let\endorigfigure\endfigure
\renewenvironment{figure}[1][2] {
    \expandafter\origfigure\expandafter[H]
} {
    \endorigfigure
}

\usepackage{amsmath} \usepackage{subfig} \usepackage{float} \renewcommand{\thesubfigure}{\Alph{subfigure}}

%% commands
%\newcommand\code{\bgroup\@makeother\_\@makeother\~\@makeother\$\@codex}
%\def\@codex#1{{\normalfont\ttfamily\hyphenchar\font=-1 #1}\egroup}
\let\code=\texttt

\let\proglang=\textsf
\newcommand{\pkg}[1]{{\fontseries{b}\selectfont #1}}

% \usepackage{color,ae,fancyvrb}

%% Sweave(-like)
\DefineVerbatimEnvironment{Sinput}{Verbatim}{fontshape=sl}
\DefineVerbatimEnvironment{Soutput}{Verbatim}{}
\DefineVerbatimEnvironment{Scode}{Verbatim}{fontshape=sl}
\newenvironment{Schunk}{}{}
\DefineVerbatimEnvironment{Code}{Verbatim}{}
\DefineVerbatimEnvironment{CodeInput}{Verbatim}{fontshape=sl}
\DefineVerbatimEnvironment{CodeOutput}{Verbatim}{}
\newenvironment{CodeChunk}{}{}
\setkeys{Gin}{width=0.8\textwidth}
%% footer

\newcommand\independent{\protect\mathpalette{\protect\independenT}{\perp}}
\def\independenT#1#2{\mathrel{\rlap{$#1#2$}\mkern2mu{#1#2}}}


\begin{document}

%\bibliographystyle{natbib}

\def\spacingset#1{\renewcommand{\baselinestretch}%
{#1}\small\normalsize} \spacingset{1}


%%%%%%%%%%%%%%%%%%%%%%%%%%%%%%%%%%%%%%%%%%%%%%%%%%%%%%%%%%%%%%%%%%%%%%%%%%%%%%

\if0\blind
{
  \title{\bf ROC and AUC with a Binary Predictor: a Potentially Misleading Metric} 
  \author{John Muschelli\thanks{This analysis was supported by NIH Grants R01NS060910 and U01NS080824 and Johns Hopkins Department of Biostatistics. }\hspace{.2cm}\\
	Department of Biostatistics, Johns Hopkins Bloomberg School of Public Health}
  \maketitle
} \fi

\if1\blind
{
  \bigskip
  \bigskip
  \bigskip
  \begin{center}
    {\LARGE\bf ROC and AUC with a Binary Predictor: a Potentially Misleading Metric}
\end{center}
  \medskip
} \fi

\bigskip
\begin{abstract}
In analysis of binary outcomes, the receiver operator characteristic
(ROC) curve is heavily used to show the performance of a model or
algorithm. The ROC curve is informative about the performance over a
series of thresholds and can be summarized by the area under the curve
(AUC), a single number. When a \textbf{predictor} is categorical, the
ROC curve has only as many thresholds as the one less than number of
categories; when the predictor is binary there is only one threshold. As
the AUC may be used in decision-making processes on determining the best
model, it important to discuss how it agrees with the intuition from the
ROC curve. We discuss how the interpolation of the curve between
thresholds with binary predictors can largely change the AUC. Overall,
we believe a linear interpolation from the ROC curve with binary
predictors, which is most commonly done in software, corresponding to
the estimated AUC. We believe these ROC curves and AUC can lead to
misleading results. We compare R, Python, Stata, and SAS software
implementations.
\end{abstract}

\noindent%
{\it Keywords:}  ROC, AUC, area under the curve
\vfill
